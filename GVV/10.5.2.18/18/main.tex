\iffalse
\let\negmedspace\undefined
\let\negthickspace\undefined
\documentclass[journal,12pt,twocolumn]{IEEEtran}
\usepackage{cite}
\usepackage{amsmath,amssymb,amsfonts}
\usepackage{graphicx}
\usepackage{textcomp}
\usepackage{xcolor}
\usepackage{txfonts}
\usepackage{listings}
\usepackage{enumitem}
\usepackage{mathtools}
\usepackage{gensymb}
\usepackage{comment}
\usepackage[breaklinks=true]{hyperref}
\usepackage{tkz-euclide} 
\usepackage{listings}
\usepackage{gvv}                                        
\def\inputGnumericTable{}                                 
\usepackage[latin1]{inputenc}                                
\usepackage{color}                                            
\usepackage{array}                                            
\usepackage{longtable}                                       
\usepackage{calc}                                             
\usepackage{multirow}                                         
\usepackage{hhline}                                           
\usepackage{ifthen}                                           
\usepackage{lscape}
\usepackage[export]{adjustbox}

\newtheorem{theorem}{Theorem}[section]
\newtheorem{problem}{Problem}
\newtheorem{proposition}{Proposition}[section]
\newtheorem{lemma}{Lemma}[section]
\newtheorem{corollary}[theorem]{Corollary}
\newtheorem{example}{Example}[section]
\newtheorem{definition}[problem]{Definition}
\newcommand{\BEQA}{\begin{eqnarray}}
\newcommand{\EEQA}{\end{eqnarray}}
\newcommand{\define}{\stackrel{\triangle}{=}}
\newtheorem{rem}{Remark}

\begin{document}
\parindent 0px
\bibliographystyle{IEEEtran}

\vspace{3cm}

\title{}
\author{EE23BTECH11042 -  Khusinadha Naik$^{*}$
}
\maketitle
\newpage
\bigskip

% \renewcommand{\thefigure}{\theenumi}
% \renewcommand{\thetable}{\theenumi}

\fi
\section*{Exercise 5.2}

\noindent \textbf{18} \hspace{2pt}The sum of the 4th and 8th terms of an AP is 24 and the sum of the 6th and 10th terms is 44. Find the first three terms of the AP.\\
\noindent \textbf{Ans.}\\

\begin{table}[h]
\centering
\begin{tabular}{|c|c|c|}
        \hline
        \textbf{Parameter} & \textbf{Value} & \textbf{Description} \\
        \hline
        $x[n]$ & ? & Input Sequence \\
        \hline
        $y[n]$ & ? & Output Sequence \\
        \hline
\end{tabular}
\caption{Input parameters table}
\label{tab:GATE.2023.BM.26.1}





\end{table}

\noindent From \tabref{tab:10.5.2.18.1}

\begin{align}
x\brak{0}+3d + x\brak{0}+7d &= 24 \label{eq:10.5.2.18.1}\\
x\brak{0}+5d + x\brak{0}+9d &= 44 \label{eq:10.5.2.18.2}
\end{align}

\noindent Subtracting \eqref{eq:10.5.2.18.1} from \eqref{eq:10.5.2.18.2}

\begin{align}
4d &= 20 \\
\implies d &= 5  \label{eq:10.5.2.18.4}
\end{align}

\noindent Putting \eqref{eq:10.5.2.18.4} in \eqref{eq:10.5.2.18.1}

\begin{align}
2x\brak{0} + 10d &= 24 \\
2x\brak{0} + 10\brak{5} &= 24 \\
\implies x\brak{0} &= -13
\end{align}

Now , general term becomes
\begin{align}
x\brak{n} &= \brak{-13 + 5n}u\brak{n}
\end{align}

Taking Z-transform of $x\brak{n}$
\begin{align}
X\brak{z} = & \frac{-13}{1 - z^{-1}} + \frac{ 5z^{-1}}{\brak{1 - z^{-1}}^2}\\
X\brak{z}\implies & \frac{18z^{-1} - 13}{z^{-2} - 2z^{-1} + 1} \quad \text{, ROC: } |z| > 1 
\end{align}

\pagebreak

Plotting $x\brak{n}$ v $n$ :
\begin{figure}[h]
    \includegraphics[width=0.5\textwidth]{ncert-maths/10/5/2/18/figs/fig.png}
    \caption{Given AP}
    \label{fig:10.5.2.18.1}
\end{figure}

From \figref{fig:10.5.2.18.1} first three terms will be
\begin{align}
\{x\brak{0},x\brak{1},x\brak{2}\} &= \{-13 , -8 , -3\}
\end{align}















%\end{document}
